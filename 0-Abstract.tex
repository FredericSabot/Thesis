\addtocounter{page}{-1}
\chapter*{Abstract}
\addcontentsline{toc}{chapter}{Abstract}
\markboth{Abstract}{Abstract}

Ensuring that power systems are always operated with a high level of reliability is becoming increasingly difficult due to increasing uncertainties coming from intermittent energy sources, market liberalisation pushing the grid closer to its limits, and the willingness of society to electrify the energy sector while building as few new power lines as possible. Transitioning from deterministic to probabilistic risk management approaches is expected to help in this endeavour as it should allow operators to run the grid closer to its limits without increasing the risk of blackouts. Probabilistic methodologies are however more complex than deterministic ones and have thus not yet reached the same level of maturity and acceptability. In particular, probabilistic security \emph{assessment} methodologies are more complex than their deterministic counterpart because they require (i) to simulate significantly more scenarios which leads to large computation times and results that are less easy to interpret, and (ii) to quantify the consequences (\eg in terms of energy not served) of the studied scenarios which requires simulating cascading outages with more accuracy than what is typically requested in deterministic studies. The objective of this thesis is to alleviate those two challenges and to develop a probabilistic dynamic security assessment methodology, \ie a security assessment methodology that considers stability issues and fast cascading outages.

The first part of this thesis focuses on the simulation of fast cascading outages. Simulating cascading outages implies to simulate the system in very degraded conditions, which challenges the models used. Most importantly, protections systems, which are typically not considered in deterministic security assessments, need to be explicitly modelled. This thesis first discusses the impact of protection systems on the initiation and propagation of cascading outages. Then, a simple indicator is proposed to predict the scenarios for which small changes in the timing of protection system operation can strongly affect the propagation of cascading outages and their final consequences. For scenarios for which this is the case, Monte Carlo simulations need to be performed to adequately assess the associated risk. Another class of models that require consideration for the simulation of cascading outages is the models of distribution grids. Indeed, distributed energy sources, especially legacy installations with limited fault ride-through capabilities, tend to disconnect themselves during severe disturbances, further degrading the state of the system. However, due to the low observability of distribution grids and the sheer number of elements connected to them, there are many uncertainties in the models of distribution grids. To handle those uncertainties, this thesis has proposed to build two dynamic equivalents per distribution grid model to bound the likely behaviour of distribution grids. These equivalents have be shown to be adequate to model cascading outages and real power systems.

The second part of this thesis deals with the larger number of scenarios that need to be considered in a probabilistic assessment and proposes a comprehensive probabilistic dynamic security assessment methodology. First, rigorous statistical accuracy indicators are proposed to determine the minimum number of scenarios that should be sampled to obtain statistically accurate results. It is then shown that a small number of critical contingencies generally contributes to a large share of the risk, reducing the number of scenarios to analyse. Moreover, simple machine-learning techniques are used to ``automatically'' identify the root causes of lack of security by identifying the boundary between secure and unsecure operating conditions for a given contingency. The proposed methodology is applied on a 73-bus system in a high-performance computing environment and the scalability to large grids is also discussed.
