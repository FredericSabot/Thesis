\chapter{Introduction}
\section{Context}

Power systems are often described by power system engineers as the largest and most complex systems built by man. Despite this complexity, they must operate without discontinuity 24 hours a day and 365 days a year. Society has grown to expect electricity to be available at any time and any interruption of service can thus have massive consequences both from an economic perspective (production downtime, equipment damage) and from a social perspective (injuries due to the lack of electricity, civil disorders).

The importance of power systems will grow even bigger with the energy transition. Indeed, to reduce their reliance on fossil fuels, countries around the globe are pushing for more electrification: petrol cars are replaced by electric cars, gas boilers are replaced by heat pumps, etc. In Europe, electricity demand is thus expected to grow by up to 53\% by 2050~\cite{ENTSOE-TYNDP2024}. In Belgium, it is even expected to grow by a factor 2 by that date~\cite{Elia-Blueprint2050}.

For this electrification to make sense, the electricity itself has to be produced by low-carbon energy sources such as wind, solar, hydro and nuclear. Apart from nuclear (that has its own set of drawbacks) and to some extent hydro, the disadvantage of these energy sources is that they are by nature weather-dependent. This makes them intermittent and makes their availability difficult to forecast, which changes the way power systems are operated. Indeed, in the past, grids were designed to follow predictable and repetitive load patterns. Conventional generators (gas, coal) would simply ramp up and down to follow the load. Nowadays, load is increasingly expected to follow generation, \ie to consume power when it is available, so when the sun is shining or when the wind is blowing.

These are thus challenging times for Transmission System Operators (TSOs) as they have to cope with (i) a rapid increase of the electricity demand, (ii) public opposition to the construction of new lines, and (iii) a change of paradigm from easily predictable load-following operations to generator-following operations with significantly more uncertainties, all of this while still fulfilling their main mission of keeping the lights on at all times.

\section{Motivation}

To guarantee that the system always stays up and running, \ie that the system is reliable,  two main types of analyses are typically performed: adequacy and security studies. A power system is said to be adequate if it has enough generation and transmission capacity to satisfy the consumers' demand at all times, even in the presence of generator or branch outages. A power system is said to be secure if it is able to withstand unscheduled faults and equipment outages without affecting consumers~\cite{AdequancySecurityDefinition}. In other words, a power system is adequate if, even with \emph{credible} asset outage(s), there always exists a system equilibrium point that satisfies all loads (and asset ratings). And, it is secure if for any \emph{credible} initial system state, the system will follow a trajectory that reaches such equilibrium after any \emph{credible} contingency.

The question that naturally arises from the above definitions is what is a \emph{credible} initial state and a \emph{credible} contingency. This equates to asking how reliable the system should be, because the more contingencies and initial states are considered, the more reliable the system will be. However, securing the system against more contingencies comes at a cost (construction of new lines, redispatches) that has to be thoroughly justified. And it is difficult to justify the construction of a new line if its only purpose is to secure the system against an unlikely contingency that might never occur in practice.

For security assessment, these questions have historically been answered by the famous ``N-1 criterion'' that states that at any moment, the system should be able to withstand the loss of any single element without consequences for end-users (no disconnection, voltages in acceptable range, etc.). And, in practice, security assessment was often performed on a small set of ``representative'' or ``umbrella states'' of the system (typically peak and minimum load), with the assumption that if the system is secure in those states, then it is always secure~\cite{CIGREreviewOfTools}.

This approach was considered appropriate for old systems dominated by dispatchable generators that followed the load. Indeed, as load patterns are quite repetitive, such was the operation of past power systems. Their security could thus adequately be assessed based on a few edge cases (peak load, minimum load). Figure~\ref{fig:belgian_dispatch} compares this to modern power systems that need to operate in a wider variety of configurations: low/high winds, low/high solar irradiation, low/high load, and any combinations of these. It is thus increasingly difficult to define a set of umbrella states that covers all weaknesses of the system, and that are at the same time reasonably likely to occur. Indeed, the probability of a given contingency occurring while the system is operating with 100\% available wind and 100\% available solar is rather slim. Designing the power system based on such state might thus be judged too conservative, and might require building more lines than desired. % On the other hand, a majority of historical blackouts and large disturbances have been caused by some form of N-k (with \(k > 1\)) contingencies~\cite{majorBlackouts, CascadingMethodoAndChallenges} which suggests that N-k contingencies should be considered (to some extent) in security assessments.

\begin{figure}
    \centering
    \begin{tikzpicture}
    \pgfplotsset{
    /pgfplots/area cycle list/.style={/pgfplots/cycle list={%
    {RedViolet,fill=red!75!black,mark=none},%
    {black,fill=black,mark=none},%
    {SeaGreen!40!white,fill=SeaGreen!40!white,mark=none},%
    {LimeGreen!80!black,fill=LimeGreen!80!black,mark=none},%
    % {pink,fill=pink,mark=none},%
    {Cerulean,fill=Cerulean,mark=none},%
    {Dandelion,fill=Dandelion,mark=none},%
    }
    },
    }
    \begin{groupplot}[
        group style={%
            group size=1 by 2,
            group name=plots,
            xlabels at=edge bottom,
            xticklabels at=edge bottom,
            vertical sep =15pt,
        },
        % footnotesize,
        % xlabel = {Time},
        ylabel = {Power [GW]},
        ybar stacked, %stack plots=y,%
        /pgf/bar width=2pt,
        area style,
        xmin=0, xmax=167,
        xtick={12,36,60,84,108,132,156},
        xticklabels={Monday, Tuesday, Wednedsay, Thursday, Friday, Saturday, Sunday},
        ymin=-1.5, ymax=12,
        tickpos=left,
        % ytick align=outside,
        % xtick align=outside,
        width=0.9\linewidth,
        height=0.3\linewidth,
    ]
    \nextgroupplot
    \addplot table [x=hour,y=nuclear] {Figs/dispatch_2016.txt};
    \addplot table [x=hour,y=other] {Figs/dispatch_2016.txt};
    \addplot table [x=hour,y=hydro] {Figs/dispatch_2016.txt};
    \addplot table [x=hour,y=gas] {Figs/dispatch_2016.txt};
    \addplot table [x=hour,y=wind] {Figs/dispatch_2016.txt};
    \addplot table [x=hour,y=solar] {Figs/dispatch_2016.txt};

    \nextgroupplot[legend to name={DispatchLegend},legend style={legend columns=6}]
    \addplot table [x=hour,y=nuclear] {Figs/dispatch_2024.txt};
    \addplot table [x=hour,y=other] {Figs/dispatch_2024.txt};
    \addplot table [x=hour,y=hydro] {Figs/dispatch_2024.txt};
    \addplot table [x=hour,y=gas] {Figs/dispatch_2024.txt};
    \addplot table [x=hour,y=wind] {Figs/dispatch_2024.txt};
    \addplot table [x=hour,y=solar] {Figs/dispatch_2024.txt};
    % \addplot table [x=hour,y=load] {Figs/dispatch_2024.txt};

    \legend{Nuclear, Coal and other solid fuels, Hydro, Gas, Wind, Solar}

    \end{groupplot}

    \node[below] at (current bounding box.south)
          {\pgfplotslegendfromname{DispatchLegend}};

    \end{tikzpicture}
    \caption{Belgian electricity mix on the week of the first of May in 2016 (above) and 2024 (below), based on data from the ENTSO-E transparency platform.}
    \label{fig:belgian_dispatch}
\end{figure}


Due to these limitations, there is an increasing willingness of TSOs and regulators to complement the \emph{deterministic} N-1 criterion with \emph{probabilistic} methods~\cite{ACER}. Probabilistic methods are based on the concept of risk. The risk associated with a given contingency is defined as the frequency of the contingency (how many times per year it is likely to occur) multiplied by the potential consequences of the contingency (\eg how many consumers will be affected if the contingency occurs). However, as the consequences of a contingency depend on the initial state of the system when the contingency occurs, the risk \(R_i\) associated with a contingency \(i\) can more rigorously be defined as

\begin{equation}
    \label{eq:risk}
    {R_i = \int f_i(x) \; p(x) \; C_i(x) \; dx}
\end{equation}
\noindent where \(f_i(x)\) is frequency of occurrence of the contingency \(i\) (given the system is currently in the state \(x\)), \(p(x)\) is the probability (density function) of the system being in the state \(x\), and \(C_i(x)\) are the consequences of contingency \(i\) if it occurs when the system is in state \(x\).

The advantage of probabilistic methods is that they allow for a better quantification of the risk of blackouts and the identification of main risk contributors. This allows the analyst to focus on those main contributors and helps in more efficiently reducing the risk. In theory, when an action is proposed to make the system more secure (construction of a new line, redispatch, etc.), it is even possible to perform a cost-benefit analysis by comparing how much the risk of blackout (and the associated costs for society) is reduced vs.\ how expensive is the proposed action. In practice however, due to inaccuracies in the risk quantification (estimation of the frequency of contingencies, evaluation of the monetary impact of blackouts, etc.), the results of such cost-benefit analysis, while suggestive, might not be sufficient on their own to justify risk-reduction actions. But while imperfect, this is still a step forward compared to deterministic methods which gives dichotomic results and limited ways to prioritise risk-reduction actions.

The drawback of probabilistic methods is that they need to consider many combinations of system states and contingencies. Their results can thus be more difficult to interpret, and they also have a higher computational burden compared to deterministic methods. Also, they require to quantify the potential consequences of contingencies (\eg how many consumers are affected) while deterministic methods only perform a dichotomic evaluation (the contingency leads to an acceptable system state or not).

Quantifying the potential consequences of a contingency is complex as it requires to simulate the behaviour of power systems in very degraded states and cascading outages. For example, let's assume that a given contingency leads the system to a state where one of its lines is overloaded. In a deterministic assessment, this state would simply be qualified as unacceptable and the simulation would stop there\footnote{Actually, it is also possible (and increasingly often done by TSOs) to simulate cascading outages in deterministic studies. It can then be accepted for a contingency to lead to a cascade, but only if it leads to no or little load shedding. However, with probabilistic approaches, more severe consequences might be accepted (if their frequency is low enough), so, longer, more complex cascades need to be simulated.}. In a probabilistic assessment, it is necessary to go further. The overloaded line could trip either following the operation of its protections or if it sags and short-circuits with the ground. Once it trips, the power flow in that line will be redirected to adjacent lines which can cause overloads in adjacent lines and/or stability issues initiating a cascading outage. One must then adequately simulate this cascading outage to see how far it spreads to estimate the final consequences of the initial contingency.

Cascading outages are particularly challenging to simulate due to the many phenomena involved and the high sensitivity of the cascading path to modelling hypotheses~\cite{CascadingMethodoAndChallenges}. Cascading outages can develop in just a few seconds or in several hours depending on the phenomena involved in a particular cascade: loss of stability leads to fast cascades, while overloads lead to slower cascades due to thermal inertia (\eg a moderately overloaded line). In recent years however, a majority of blackouts have been caused by fast cascading outages lasting a few seconds to a few minutes~\cite{cascadeAcceleration}. This is a side effect of the increasing importance of stability issues in power grids. This increase is caused by many factors including reduced system inertia with the introduction of inverter-based generation (wind, solar), high-distance power transfers from renewable-rich regions to load centres, market liberalisation pushing the grid closer to its limits, and increase of static limits through the use of dynamic line rating and better conductors.

However, only a small share of the literature on cascading outages considers fast cascades. And, research works that consider fast cascading outages suffer from high computational requirements as they need to perform many detailed time-domain simulations for each given cascade to assess the impact of modelling uncertainties on the cascading path. They thus cannot be used to perform probabilistic security assessment of large power systems as probabilistic security assessment of systems with many possible system states and contingencies are already computationally expensive.


\section{Objectives and outline of the thesis}

The objective of this thesis is thus to develop a probabilistic security assessment methodology that adequately models stability issues and fast cascading outages, yet is computationally efficient enough to be applied to large power systems. Also, the method should give intuitive results and recommendations on how to cost-effectively reduce the risk of cascading outages. Due to the high computational cost of probabilistic security assessments, this thesis focuses on planning timescales (at least one year ahead) where speed is less critical, although probabilistic security assessment methodologies could also be used in operational planning (days to months ahead).

But before building such methodology, it is necessary to have a good understanding of the current state-of-the-art in security assessment methodologies. Thus,

\begin{itemize}
    \item Chapter~\ref{ch:security} introduces in more detail security assessment and deterministic methodologies, reviews the literature in probabilistic methodologies, and discusses the achievements and gaps in existing works.
\end{itemize}

This thesis then splits into two parts. Part~\ref{part:models} tackles the problem of simulating power systems in very degraded states and fast cascading outages because, as discussed above, this is a key prerequisite to build a probabilistic security assessment methodology, especially when considering stability issues. More specifically,

\begin{itemize}
    \item Chapter~\ref{ch:protections} introduces the topic of power system protection as protections play a key role in the triggering and evolution of cascading outages. It then describes their failure modes and how they affect security. Finally, it proposes a method to handle the high sensitivity of the cascading processes to the timing of protection system operations. Annex~\ref{ch:SPS} complements this chapter by giving some reliability considerations regarding system integrity protection schemes, a particular kind of protection systems aimed at improving power system security.
    % \item Chapter~\ref{ch:SPS} focuses on the special kind of protection systems that are System Integrity Protection Schemes (SIPSs). As opposed to traditional protections, SIPSs do not protect individual elements of the grid (\eg against overloads), but automatically perform corrective actions with the aim of maintaining the stability of the whole grid. They can thus be a cost-effective approach to reduce the risk of blackouts.
    \item Chapter~\ref{ch:distrib} discusses how to model the impact of distribution grids on transmission system stability. Indeed, distribution systems are playing a growing role in the dynamic behaviour of power systems due to the growing share of distributed energy resources (DERs, such as rooftop solar generation). This is especially true during degraded system conditions, as DERs are prone to disconnect themselves during disturbances, adding stress on an already weakened system.
\end{itemize}

With this, part~\ref{part:PDSA} can then build a probabilistic dynamic security assessment methodology, and

\begin{itemize}
    \item Chapter~\ref{ch:DPSA} proposes such a methodology for probabilistic dynamic security assessment. It uses the models developed in the previous chapters, and efficient sampling and screening techniques to build a methodology that is accurate and can be applied to large power systems with high but manageable computational resources. Also, it shows how the methodology can identify the main contributors to the risk and propose recommendations on how to best reduce the risk.
\end{itemize}

Finally, chapter~\ref{ch:perspectives} concludes with a summary and some perspectives.

\section{List of publications}

The work presented in this thesis has led to the following publications.

\begin{itemize}
    \item \fullcite{Journal_paper}
    \item \fullcite{CIGRE_paper}
    \item \fullcite{Cosim}
    \item \fullcite{ISGT2023_ADN}
    \item \fullcite{ISGT2023_Protections}
    \item \fullcite{LambdaMu2022}
    \item \fullcite{MCDETasTool}
\end{itemize}

The data and code used in these publications is publicly available at~\url{https://fredericsabot.github.io/Publications.html}.

This thesis was performed in the framework of the CYPRESS project (\url{https://cypress-project.be/}). I co-authored the following deliverables.

\begin{itemize}
    \item \fullcite{cypressD11}
    \item \fullcite{cypressD13}
    \item \fullcite{cypressD21}
\end{itemize}
